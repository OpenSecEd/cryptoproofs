\title{Cryptographic Proofs}
\subtitle{%
  Sequences of games
}
\author[D. Bosk <dbosk@kth.se>]{Daniel Bosk}
\institute[KTH CSC]{%
  School of Computer Science and Communication\\
  KTH Royal Institute of Technology\\
  \email{dbosk@kth.se}
}
\date{22nd January 2016}

\begin{frame}
  \maketitle
\end{frame}

\mode*

\begin{abstract}
  \input{abstract.tex}
\end{abstract}


\section{Introduction}

\subsection{What to prove?}

\begin{frame}
  \begin{itemize}

    \item When designing primitives we need to prove their security.
      \begin{itemize}
        \item We formalize the properties we want.
        \item Then prove our model has these properties.
      \end{itemize}

      \pause{}

    \item Sometimes this can be done as a security game.

  \end{itemize}
\end{frame}

\begin{frame}
  \begin{example}[IND-CPA]
    Public-key encryption scheme \((\KeygenOp, \EncOp, \DecOp)\).
    \begin{enumerate}
        
        \pause{}

      \item \((\PubKey{}, \PriKey{})\gets \Keygen{}\), give \(\PubKey{}\) to 
        adversary \(A\).

        \pause{}

      \item \(A\) chooses \(m_0, m_1\in M\) and gives back to challenger.

        \pause{}

      \item Challenger computes \(b\rgets \{0, 1\}, \psi\rgets \Enc[\PubKey{}]{ 
            m_b }\), gives \(psi\) to \(A\).

        \pause{}

      \item \(A\) computes \(\hat{b}\), if \(\hat{b} = b\) the adversary wins 
        the game.

    \end{enumerate}

    \pause{}

    Let the IND-CPA advantage of \(A\) be \(|\Pr( b = \hat{b} 
      ) - \frac{1}{2}|\) be the adversary's advantage.
  \end{example}
\end{frame}

\subsection{Two approaches}

\begin{frame}
  \begin{itemize}
      \item There are basically two approaches to this.
        \begin{itemize}
          \item That of Shoup~\cite{Shoup2004Sequences}.
          \item That of Bellare and Rogaway~\cite{Bellare2004CodeBased}.
        \end{itemize}

        \pause{}

      \item They are essentially equivalent.

        \pause{}

      \item Previous example was Shoup style.

      \item Next is the same Bellare-Rogaway style.
  \end{itemize}
\end{frame}

\begin{frame}
  \begin{example}[IND-CPA]
    \begin{algorithmic}
      \Function{Initialize}{}
        \State{$(\PubKey{},\PriKey{})\rgets \Keygen{}, b\rgets \{0, 1\}$}
        \State{\Return{$\PubKey{}$}}
      \EndFunction{}

      \pause{}

      \Statex{}

      \Function{LR}{$m_0, m_1$}
        \State{$c\rgets \Enc[\PriKey{}]{ m_b }$, \Return{$c$}}
      \EndFunction{}

      \pause{}
      \Statex{}

      \Function{Deinitialize}{$\hat{b}$}
        \State{\Return{$(\hat{b} = b)$}}
      \EndFunction{}
    \end{algorithmic}

    \pause{}

    Let the IND-CPA advantage of \(A\) be \(|\Pr( b = \hat{b} 
      ) - \frac{1}{2}|\) be the adversary's advantage.
  \end{example}
\end{frame}


\section{Sequences of Games}

\subsection{The basic idea}

\begin{frame}
  \begin{block}{Players}
    \begin{itemize}
      \item Attack game played between adversary and challenger.
      \item Both probabilistic processes communicating with each other.
    \end{itemize}
  \end{block}

  \pause{}

  \begin{block}{Security}
    \begin{itemize}
      \item Event \(S\).
      \item For every \emph{polynomial-time} adversary \dots
      \item \dots \(\Pr(S)\) is \emph{negligibly close to} some \enquote{target 
          probability}.
    \end{itemize}
  \end{block}

\end{frame}

\begin{frame}
  \begin{block}{Proving}
    \begin{itemize}
      \item Let Game 0 be the original game (adversary versus primitive),
        \(S_0 = S\).

        \pause{}

      \item Construct Game 1 with \(S_1\) such that \(|\Pr(S_0) - \Pr(S_1)|\) 
        is negligible.

        \pause{}

      \item \dots

      \item Construct Game \(n\) with \(S_n\) such that \(\Pr(S_n)\) is the 
        target probability.

        \pause{}

      \item \color{red} Since \(n\) is constant, it follows that \(\Pr(S)\) is 
        negligibly close to the target probability.
    \end{itemize}
  \end{block}
\end{frame}

\begin{frame}
  \begin{question}
    \begin{enumerate}
      \item Why is this?
      \item Is it because we don't know if the (countably) infinite sum of 
        negligible differences is negligible in a polynomial security 
        parameter?
    \end{enumerate}
  \end{question}
\end{frame}

\begin{frame}
  \begin{itemize}
    \item If changes between games are small.
    \item Then we can analyse them better.
  \end{itemize}
\end{frame}

\subsection{Techniques for transitioning between games}

\begin{frame}
  \begin{itemize}
    \item Transitions based on indistinguishability

    \item Transitions based on failure events

    \item Bridging steps
  \end{itemize}
\end{frame}

\begin{frame}
  \begin{block}{Indistinguishability}
    \begin{itemize}
      \item The Hybrid Argument
      \item Show that if the adversary can win, then the adversary can 
        distinguish two indistinguishable distributions.

        \pause{}

      \item \color{red} Show that \(|\Pr(S_i) - \Pr(S_{i+1})|\) is negligible.
    \end{itemize}
  \end{block}

  \begin{question}
    Is that correct?
  \end{question}
\end{frame}

\begin{frame}
  \begin{block}{Failure event}
    \begin{itemize}

      \item \(F\) is sometimes a security assumption:
        \begin{itemize}
          \item Adversary found collision in hash function.
          \item Adversary forged \iac{MAC}.
        \end{itemize}

        \pause{}

      \item Sometimes purely information theoretic argument.

      \item Sometimes \(F\) requires its own sequence of games.

        \pause{}

      \item \color{red} Show that \(|\Pr(S_i) - \Pr(S_{i+1})|\leq \Pr(F)\), and 
        we might \enquote{know} that \(\Pr(F)\) is negligible.

    \end{itemize}
  \end{block}
\end{frame}

\begin{frame}
  \begin{lemma}[Difference Lemma]
    \begin{itemize}
      \item \(A, B, F\) events in some distribution.
      \item Suppose \(A\land \lnot F\iff B\land \lnot F\).
      \item Then \(|\Pr(A) - \Pr(B)|\leq \Pr(F)\).
    \end{itemize}
  \end{lemma}
\end{frame}

\begin{frame}
  \begin{proof}
    \small
    \begin{align*}
      |\Pr(A) - \Pr(B)|
      &= |\Pr(A\land F) + \Pr(A\land \lnot F)
      - \Pr(B\land F) - \Pr(B\land \lnot F)| \\
      &= \color{blue} [\text{assumption } \Pr(A\land\lnot F) = \Pr(B\land\lnot 
      F)] \\
      &= |\Pr(A\land F) - \Pr(B\land F)| \\
      &\leq \color{blue} [ \Pr(A\land F)\leq \Pr(F), \Pr(B\land F)\leq \Pr(F) 
      ] \\
      &\leq \Pr(F).
    \end{align*}
  \end{proof}
\end{frame}

\begin{frame}
  \begin{block}{Bridging step}
    \begin{itemize}
      \item This is simply a transformation and we prove their equivalence.
      \item This is useful for making the proof easier to read.

        \pause{}

      \item \color{red} Show that \(\Pr(S_i) = \Pr(S_{i+1})\).
    \end{itemize}
  \end{block}
\end{frame}

\subsection{An example of ElGamal encryption}

\begin{frame}
  \begin{example}[ElGamal scheme]
    Group \(G\) of prime order \(q\), \(\langle\gamma\rangle = G\).
    \begin{description}
      \item[Keygen] \(x\rgets \Z_q\) then
        \(\alpha\gets \gamma^x, \PubKey{}\gets \alpha, \PriKey{}\gets x\).

      \item[Enc] \(m\in G\), then \(y\rgets \Z_q, \beta\gets \gamma^y, 
          \delta\gets \alpha^y\) and
        \(\zeta\gets \delta\cdot m, \psi\gets (\beta, \zeta)\).

      \item[Dec] \(m\gets \zeta/\beta^x = \alpha^y m/\beta^x = (\gamma^x)^y 
          m/(\gamma^y)^x = \gamma^{xy}m/\gamma^{xy} = m\).
    \end{description}
  \end{example}
\end{frame}

\begin{frame}
  \begin{itemize}
    \item ElGamal is semantically secure (IND-CPA) under \ac{DDH}.
    \item Thus we plug ElGamal into the IND-CPA game from above.
  \end{itemize}
\end{frame}

\begin{frame}
  \begin{example}[ElGamal in IND-CPA, Game 0]
    \begin{enumerate}
      \item \(x\rgets \Z_q, \alpha\gets\gamma^x\)
      \item \(r\rgets R, (m_0, m_1)\gets A(r, \alpha)\)
      \item \(b\rgets \{0,1\}, y\rgets \Z_q,
          {\color{blue} \beta\gets \gamma^y, \delta\gets \alpha^y,
          \zeta\gets \delta\cdot m_b}\)
      \item \(\hat{b}\gets A(r,
          \alpha = \gamma^x, \beta = \gamma^y, \zeta = \gamma^{xy}\cdot m_b)\)
    \end{enumerate}
    \(S_0\) is event \(b = \hat{b}\), thus advantage is \(|\Pr(S_0) 
      - \frac{1}{2}|\).
  \end{example}
\end{frame}

\begin{frame}
  \begin{example}[DDH in IND-CPA, Game 1]
    \begin{enumerate}
      \item \(x\rgets \Z_q, \alpha\gets\gamma^x\)
      \item \(r\rgets R, (m_0, m_1)\gets A(r, \alpha)\)
      \item \(b\rgets \{0,1\}, y\rgets \Z_q,
          {\color{blue} \beta\gets \gamma^y,
            {\color{red} z\rgets \Z_q, \delta\gets \gamma^z},
          \zeta\gets \delta\cdot m_b}\)
      \item \(\hat{b}\gets A(r,
          \alpha = \gamma^x, \beta = \gamma^y, \zeta = {\color{red} \gamma^z} 
          \cdot m_b)\)
    \end{enumerate}
    \(S_1\) is event \(b = \hat{b}\), thus advantage is \(|\Pr(S_1) 
      - \frac{1}{2}|\).
  \end{example}
\end{frame}

\begin{frame}
  \begin{block}{Claim 1}
    \(\Pr(S_1) = \frac{1}{2}\), i.e.\ no better than random guessing.
  \end{block}

  \begin{proof}[Proof sketch]
    \begin{itemize}
      \item \(\delta\) in Game 1 is a one-time pad.
      \item Thus \(\hat{b}\) is independent from \(b\).
    \end{itemize}
  \end{proof}
\end{frame}

\begin{frame}
  \begin{block}{Claim 2}
    \(|\Pr(S_0) - \Pr(S_1)| = \epsilon_{DDH}\), i.e.\ negligible under 
    \ac{DDH}.
  \end{block}

  \begin{proof}[Proof sketch]
    \begin{itemize}
      \item Game 0: \((\gamma^x, \gamma^y, \gamma^{xy})\),
        Game 1: \((\gamma^x, \gamma^y, \gamma^z)\).

      \item Define algorithm \(D\) that uses \(A\) to distinguish between Game 
        0 and Game 1.

        \pause{}

        \begin{itemize}
          \item \(\Pr[ x, y\rgets \Z_q : D(\gamma^x, \gamma^y, \gamma^{xy} 
              ) = 1 ] = \Pr(S_0)\)

          \item \(\Pr[ x, y, z\rgets Z_q : D(\gamma^x, \gamma^y, \gamma^z) 
              = 1 ] = \Pr(S_1)\)
        \end{itemize}

      \item \color{red} Thus \ac{DDH} advantage of \(D\) is \(|\Pr(S_0) 
          - \Pr(S_1)|\).
    \end{itemize}
  \end{proof}
\end{frame}

\begin{frame}
  \begin{question}
    Is that because the advantage in \ac{DDH} is
    \begin{multline*}
      |\overbrace{\Pr[ x, y\rgets \Z_q : D(\gamma^x, \gamma^y, \gamma^{xy} 
        ) = 1 ]}^{\Pr(S_0)}\\
      - \underbrace{\Pr[ x, y, z\rgets Z_q : D(\gamma^x, \gamma^y, \gamma^z) 
        = 1 ]}_{\Pr(S_1)}|
    \end{multline*}
  \end{question}
\end{frame}

\begin{frame}
  \begin{block}{Algorithm}
    \begin{algorithmic}
      \Function{D}{$\alpha, \beta, \delta$}
        \State{$r\rgets R, (m_0, m_1)\gets A(r, \alpha)$}
        %\Comment{Use \(A\) that breaks ElGamal}
        \State{$b\rgets \{0,1\}, \zeta\gets \delta\cdot m_b$}
        \State{$\hat{b}\gets A(r, \alpha, \beta, \zeta)$}
        %\Comment{Use \(A\) to break DDH}
        \If{$b = \hat{b}$}
          \State{\Return{1}}
        \Else{}
          \State{\Return{0}}
        \EndIf{}
      \EndFunction{}
    \end{algorithmic}
  \end{block}
\end{frame}

\begin{frame}
  \begin{proof}[Proof ElGamal is IND-CPA]
    Combine Claim 1 and Claim 2.
  \end{proof}
\end{frame}

% XXX Add examples from AuthEncryption
%\subsection{An example from Authenticated Encryption}
%
%% XXX Add example of a failure event
%\begin{frame}
%  \begin{example}[Failure event]
%    \dots
%  \end{example}
%\end{frame}


\section{Discussion}

\subsection{Comments}

\begin{frame}
  \begin{itemize}
    \item I like Shoup's clarity concerning the adversary algorithm \(A\).
    \item However, Bellare has (many) easy real-world examples in a few papers.
  \end{itemize}
\end{frame}

\begin{frame}
  \begin{block}{Recommended reading}
    \begin{itemize}
      \item \fullcite{AuthEncryption}
      \item \fullcite{StatefulDecryption}
    \end{itemize}
  \end{block}
\end{frame}

\subsection{Questions}

\begin{frame}
  \begin{block}{Statement}
    \begin{itemize}
      \item Best if two successive games are defined on the same underlying 
        probability space.

      \item Good practice to have all games in a sequence defined on the same 
        probability space.
    \end{itemize}
  \end{block}

  \begin{definition}[Probability space]
    \begin{itemize}
      \item Sample space \(\Omega\), set of all outcomes.
      \item Set of events \(E\), each event contains one of more outcomes.
      \item \(\Pr\colon E\to [0, 1]\).
    \end{itemize}
  \end{definition}
\end{frame}

\begin{frame}
  \begin{question}
    \begin{enumerate}
      \item How do we determine the probability space of a game?
        The players?

      \item\label{ExampleDifferentSpaces} An example of two games that are 
        defined on different probability spaces?
    \end{enumerate}
  \end{question}

  \begin{example}[Question~\ref{ExampleDifferentSpaces}]
    \begin{itemize}
      \item Events \(A, B, F_1, F_2\).
      \item \(\Pr(A\land \lnot F_1) = \Pr(B\land \lnot F_2)\) and \(\Pr(F_1) 
          = \Pr(F_2)\).
      \item Then \(|\Pr(A) - \Pr(B)|\leq \Pr(F_1) = \Pr(F_2)\).
    \end{itemize}
  \end{example}
\end{frame}

\begin{frame}
  \begin{example}[Hashed ElGamal]
    \begin{enumerate}
      \item Hashed ElGamal in IND-CPA game
      \item\label{ElGamalToDDH} Transform ElGamal to DDH
      \item\label{HashToRandom} Transform hash value to random value.
    \end{enumerate}
  \end{example}

  \begin{question}
    \begin{itemize}
      \item Does the order of step~\ref{ElGamalToDDH} and~\ref{HashToRandom} 
        matter?
      \item Or is it just about what's easiest to follow?
    \end{itemize}
  \end{question}
\end{frame}

\begin{frame}
  \begin{question}
    Any other thoughts, comments or questions?
  \end{question}
\end{frame}


%%% REFERENCES %%%

\begin{frame}[allowframebreaks]
  \printbibliography{}
\end{frame}
